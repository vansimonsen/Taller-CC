\documentclass[fleqn]{article}
\usepackage[utf8]{inputenc}
\usepackage{fancyvrb}
\usepackage{fancyhdr}
\usepackage{url}
\usepackage{verbatim}
\usepackage{graphicx}
\usepackage{rotating}
\usepackage{color}
\usepackage{amsmath}
\usepackage{mathtools}
\usepackage{listings}

\parskip 1mm
\setlength{\topmargin}{0pt}
\oddsidemargin  0.5cm
\evensidemargin 0.5cm
\textwidth      15.5cm
\textheight     21.0cm
\headsep        4 mm
\parindent      0.5cm

\pagestyle{fancyplain}

\lhead{Taller de modelos y métodos 2015-2}
\rhead{\bf \it Primer Informe}
\lfoot{}
\cfoot{}
\rfoot{\bf \thepage}
\renewcommand{\footrulewidth}{0.4pt}

\title{\textbf{Taller de módelos y métodos}\\Aplicación de precondicionadores}
\author{Ariel Sanhueza - \textit{asanhuez@alumnos.inf.utfsm.cl} - \textit{201173005-4}\\
{Axel Símonsen - \textit{axel.simonsen@alumnos.usm.cl} - \textit{201173007-0}}}


\date{\vspace*{1cm} Valparaíso, \today}

\begin{document}
\maketitle

\begin{abstract}
Al resolver el sistema de ecuaciones de la forma $A\vec{x}=\vec{b}$, dependemos en gran medida de la calidad de la matriz. Cuando se da el caso de que la matriz está mal condicionada, el error en la resolución del sistema de ecuaciones crece. Cuando esto ocurre, necesitamos encontrar un modo de mejorar nuestra matriz con el objetivo de disminuir su número de condición. A esta solución se le llama \emph{Precondicionador}, el cual es una matriz aplicada a la matriz A, que mejora su número de condición.  
\end{abstract}

\section{Desarrollo del problema}

Sea un sistema de ecuaciones de la forma $Ax = b$, con $A \in \mathcal{C}_{nxn}, x, b \in \mathcal{C}_{nx1}$, donde $\mathcal{K}(A) >> 1$, es decir, la matriz está mal condicionada.

El parámetro presentado para medir la eficiencia de GMRes es un valor $\rho$ definido como:

\begin{align*}
\rho &= \frac{\lambda_M - \lambda_m + 2\delta}{\lambda_M + \lambda_m + 2\sqrt{\lambda_M \lambda_m + \delta^2}}
\end{align*}

Donde:\\
$\lambda_{M}$ = mayor $Re(\lambda)$\\
$\lambda_{m}$ = menor $Re(\lambda)$\\
$\delta$ = mayor $Im(\lambda)$\\
\\

El problema mencionado actualmente propone encontrar una matriz M, la cual se define como:


\begin{equation}
 M= \alpha I + A 
\end{equation}

\newpage
Para el problema planteado en el paper, se buscan los $\alpha$ tal que la siguiente expresión sea minimizada dado que se requiere minimizar la tasa de convergencia para el problema $f(M)A\vec{x} = f(M)\vec{b}$, donde $f(M)$ es el precondicionador actualmente propuesto, el cual se explicará mas adelante:

\begin{equation}
 \rho_{f(M)A}(\alpha) = \frac{\lambda_{M}(\lambda_{M}+\alpha)^{-1} - \lambda_{m}(\lambda_{m}+\alpha)^{-1} + 2\delta}{\lambda_{M}(\lambda_{M}+\alpha)^{-1} + \lambda_{m}(\lambda_{m}+\alpha)^{-1} + 2\alpha + 2\sqrt{(\lambda_{M}(\lambda_{M}+\alpha)^{-1}+\alpha)(\lambda_{m}(\lambda_{m}+\alpha)^{-1}+\alpha) + \delta^{2}}}\\
\end{equation}

La idea es encontrar un $\alpha$ que minimice esta expresión, tal que $\rho_{f(M)A}(\alpha) \le \rho$, con el objetivo de disminuir la tasa de convergencia de GMRes (método utilizado en el \textit{paper} para resolver el sistema). Sin embargo, el parámetro no puede ser muy pequeño, pues se asemeja a no hacer nada, y no puede ser muy grande, dado que no crearíamos un buen precondicionador.

Se propone actualmente una matriz M, cuya inversa sea cercana a la inversa de A, con el objetivo de llegar a algo similar a la identidad, sin tener que calcular la inversa de A, cuyo costo computacional es alto.

El problema presentado original propone como precondicionador la evaluación de la matriz $M$ en la función $f(x) = \frac{1}{x}$. Esta función matricial es definida por medio de la fórmula integral de Cauchy:

\begin{align*}
f(M) &= \frac{1}{2\pi i} \oint_C f(z)(zI - M)^{-1}dz
\end{align*}

donde $f(z)$ es una función analítica (en el problema presentado, vale decir que sea representable como una serie de potencias convergente) y $f(x) = x^{-1}$, $C$ una elipse que encierre todos los valores propios de la matriz $M$. Esta definición nos permite saber el valor de una función matricial sin la necesidad de evaluar \textit{directamente} la matriz en la función y se reduce finalmente al cálculo de una integral de línea cerrada sobre la curva $C$.

A pesar de la definición previa, nosotros no queremos computar explícitamente $f(M)$, si no que $f(M)\vec{v}$, lo que queda finalmente:
\begin{align*}
f(M)\vec{v} &= \frac{1}{2\pi i} \oint_C f(z)(zI - M)^{-1}\vec{v}dz
\end{align*}

que se reduce finalmente a resolver el sistema de ecuaciones $(Iz - M)\vec{h} = \vec{v}$ por medio de GMres para obtener $\vec{h} = (zI - M)^{-1}\vec{v}$. La integral finalmente se discretiza para poder generar una aproximación de su valor (por ejemplo, utilizando el método del trapezoide).

Nuestro trabajo es buscar otro precondicionador, diferente al propuesto, que permita mejorar aun más el cálculo del sistema de ecuaciones. Se buscará otra función analítica $f(x)$ que no solo mejore el número de condición de la matriz $A$, si no que también sea mejor que el precondicionador propuesto.\newpage

\section{Plan de trabajo:}

Se propone como primer objetivo encontrar una función que mejore la ya propuesta en el paper, y a partir de esta construir un algoritmo que dada una matriz mal condicionada, encuentre el parámetro $\alpha$ para construir M y construya la matriz $f(M)$, con la cual se pueda mejorar la resolución del sistema $A\vec{x} = \vec{b}$, transformándolo en el sistema $f(M)A\vec{x} = f(M)\vec{b}$. Para medir que tan bueno es nuestro precondicionador, utilizaremos los valores propios de la matriz original, aplicándoles la función seleccionada, y con ellos calculando el número de condición, el cual servirá como parámetro de medida de la eficacia del precondicionador utilizado, junto con el calculo del parametro $\rho$, para el precondicionador seleccionado por la matriz ingresada. Todo lo anterior estará apoyado en los papers sugeridos e investigación independiente. 

\end{document}
